\newpage
\section{Úvod}
Pro konfiguraci EPL sítě existuje nástroj OpenConfigurator.
Ten však nezvládá některé námi požadované operace a spolupráci se systémem REX.
Tento nástroj je napsán v jazyce Tcl, což činí jeho další rozšíření komplikovanější.
Proto bylo rozhodnuto o napsání nového konfigurátoru sítě Ethernet POWERLINK, 
který byl dobře rozšiřitelný a uměl spolupracovat se systémem REX.

%
% Keywords
%
\section{Pojmy}
\subsection{Nod}
Objekt který reprezentuje jeden uzel EPL sítě. 
Nod má určité vlastnosti. 
Dále obsahuje indexy, subindexy a domény, které jsou uspořádány ve stromové struktuře.
Definovaný bývá zpravidla XDD souborem.

\subsection{Connection}
Mapování objektů (index, subindex, doména) mezi jednotlivými nody tak, jak je definováno v EPL.
PDO odpovídají vztahu producer/consumer, existuje vždy jeden zdroj dat avšak více odběratelů. 
V connection je tedy vždy jeden zdroj dat a jedne nebo více odběratelů. 
Connection dále obsahuje identifikační číslo, název a typ, který je definován typem consumera.

\subsection{Alias}
Zástupný název pro index, subindex nebo doménu.

\subsection{Projekt}
Objekt, který obsahuje nody, jejich propojení (connection) a aliasy.
Dále má projekt určité vlastnosti a parametry pro generování mapování.

%
% Menu description
%
\section{Menu}
\subsection{Project}
\begin{itemize}
\item New vytvoří nový projekt.
\item Open načte existující projekt. Uživatel je pomocí dialogu vyzván k tomu, aby projekt vybral. 
\item Close zavře otevřený projekt.
\item Save uloží projekt do souboru, ze kterého byl načten.
\item Save as uloží projekt do nového souboru. Soubor je vybrán pomocí dialogu.
\item Recent projects zobrazí seznam několika posledních otevřených projektů.
\item Reload znovu načte projekt ze souboru.
\item Exit uzavře projekt. 
\end{itemize}

\subsection{Network}
\begin{itemize}
\item Add node přidá nový node do projektu. Uživatel je vyzván, aby vybral soubor obsahující popis nodu.
\item Network properties \ldots
\end{itemize}

\subsection{Configuration}
\begin{itemize}
\item Generate PDO automaticky generuje PDO mapování.
\item Generate output for REX automaticky generuje nastavení pro systém REX.
\item Open folder otevře adresář, ve kterém se nachází právě otevřený projekt.
\end{itemize}

\subsection{Help}
\begin{itemize}
\item About EplConfig zobrazí stručné informace o programu.
\end{itemize}

%
% Function description
%
\section{Funkce}
Pomocí EplConfig lze provést kompletní konfiguraci sítě Ethernet POWERLINK, 
do které jsou připojeny nódy se systémem REX i nódy třetích stran. 

\subsection{Přidání nodu}
Pro každý nód v síti je importován soubor xdd. Při importu se EplConfig ptá na
\begin{itemize}
\item soubor XDD (libovolný zvolený uživatelem nebo výchozí pro REXový nód),
\item index nódu v síti (240 pro MN), 1 \ldots 239 pro CN,
\item název nódu,
\item infromaci o tom, jestli je nod REX-nód.
\end{itemize}

\subsection{Editace hodnot object dictionary}
Konfigurace je vytvořena na základě importovaných xdd souborů po uživatelské editaci.

EplConfig zpřístupní hodnoty, které mají nastaven přístup na "w", "rw" pro editaci. 
Při editaci je nutné respektovat typ a formát dat.
Struktura projektu je zobrazena ve stromové hierarchii. 
Úroveň ve stromové hierarchii je následující:
\begin{itemize}
\item Název projektu,
\item název a číslo nodu,
\item číslo a název indexu,
\item číslo a název subindexu,
\item pokud je subindex binární pole (domain) a je k dispozici popis tohoto pole, 
pak jsou zde položky tohoto pole, jinak 5. úroveň není.
\end{itemize}

U každého editovatelného objektu (indexu, subindex) je možné zakliknout, 
zda bude objekt zahrnut do výstupních souborů pro další použití v REXu.
Vytváření objektů v OD pro REXové nódy
Manufacturer specific area (2000h - 5FFFh) bude v rexu generována dynamicky 
na základě funkčního algoritmu a požadavků na vstupy a výstupy. 
EplConfig musí umožnit vytvářet tuto oblast tj. 
je možné přidávat objekty (indexy, subindexy) do REXových nódů.

Objekty lze vytvářet těmito způsoby:

\begin{itemize}
\item Ručně. Uživatel sám zvolí číslo a název indexu a subindexu, typ dat a výchozí hodnotu.
\item Automaticky na základě mapovaní PDO (viz dále).
\end{itemize}

\subsection{Vytváření PDO mapování}
EplConfig umožňuje vytvářet datové propojení (connection) mezi nódy. 
Cíl i odběratel je jednoznačně identifikován ve tvaru nód.index.subindex[.offset v binárním poli]. 

Uživateli je tento tvar zobrazen v popisné textové podobě, uvnitř EplConfig jsou uloženy číselné hodnoty. 
Není možné, aby byly v jedné connection byl více než jeden odkaz na data z jednoho nódu!

Vytvoření nové connection.
\begin{itemize}
\item V pravém panelu zvolíme záložku Connection. 
Ve stromu projektu zvolíme objekt a přetáhneme ho na prázdný řádek connection.
\item Pravým tlačítkem klikneme na tabulku mapování a zvolíme Add connection.
\end{itemize}

Editace mapování.
\begin{itemize}
\item Změna mapování. Přetáhneme objekt ze stromu a přetáhneme ho na požadované místo v tabulce.
\item Odstanění prvku. Zvolíme danou buňku v tabulce a stisknutím klávesy Delete odebereme prvek. 
Při označení identifikátoru, nebo názvu mapování se odstraní celé mapování.
\item Přejmenování mapování zvolíme název, který chceme změnit a stiskneme klávesu F2, nebo dvakrát klikneme.
Objeví se editační buňka, kde upravíme název connection.
\end{itemize}

\subsection{Vytváření aliasů pro REXové nódy}
Ve schématu systému REX se na data okazuje vstupním nebo výstupním blokem pojmenovaným ve tvaru EPL OdkazNaData.
OdkazNaData musí jednoznačně odkazovat na data. 
Pokud se OdkazNaData shoduje s názvem subindexu, jedná se o odkaz na tento subindex. 
EPL však umožňuje pracovat s daty typu binární pole (domain). 
Pokud je v OD vytvořen objekt tohoto typu, je nutné umožnit přístup i k offsetu v tomto poli. 
Proto jsou zavedeny datové aliasy. 
Datový alias jednoznačně identifikuje data pro daný nód ve tvaru index.subindex(.offset). 
Uživateli je alias opět zobrazen v textové podobě, v programu je odkaz uložen číselně.

Aliasy lze editovat v pravém panelu v záložce Aliases.
\begin{itemize}
\item Vytvoření aliasu proběhne
\subitem editací jména aliasu na prázdném řádku, klávesou F2, nebo dvojitým kliknutím, 
\subitem přetažením objektu ze stromu na prázdný řádek.
\item Upravení aliasu
\subitem editací jména aliasu na příslušném řádku, klávesou F2, nebo dvojitým kliknutím, 
\subitem přetažením objektu ze stromu na příslušný řádek.
\item Odstranění aliasu proběhne označením příslušného řádku a stisknutí klávesy Delete.
\end{itemize}

\subsection{Generování PDO mapování}
Na základě vytvořeného PDO mapování jsou vygenerovány hodnoty pro objekty 1400h, 1600h, 1800h, 1A00h. 
Při generování je nutné provést typovou kontrolu a brát zřetel na fakt, 
že pro některé nódy mohou být hodnoty v těchto objektech pevně dané tj. read-only. 
V takovém případě je možné manipulovat pouze s hodnotami "na druhé straně".

\subsection{Generování výstupu pro RexView}
Výstup z EplConfig je předán RexView v textové podobě ve formátu mdl. Soubor obsahuje:
\begin{itemize}
\item Seznam vytvořených objektů pro každý nód (podle odstavce 2).
\item Seznam vytvořených aliasů pro každý nód.
\item Výchozí nastavení OD tj. seznam hodnot ve tvaru index,subindex,data pro každý nód. 
Tento seznam bude zahrnovat veškeré PDO mapování (tj. objekty 1400h, 1600h, 1800h, 1A00h) a objekty, 
které jsou zahrnuté do generovaného výstupu.
\end{itemize}

